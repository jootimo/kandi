\setlength{\hoffset}{-1in} \setlength{\oddsidemargin}{4cm} \addtolength{\textwidth}{1.3cm} \addtolength{\textheight}{1cm} \setlength{\voffset}{-1in}
\thispagestyle{empty}  %ei sivunumeroa sivun alareunaan

\noindent
TURUN YLIOPISTO\\
Informaatioteknologian laitos\\
\\
HEINONEN, TIMO: Avaruusjakoon perustuvat tietorakenteet tietokonegrafiikassa\\
kandidaatintutkielma, \pageref{LastPage} s.\\
Tietojenkäsittelytieteet\\
\today\\
\rule{\textwidth}{.2mm}\\
\\
Tässä tutkielmassa tutustutaan kolmiulotteisten kuvien hahmontamiseen kaksiulotteisiksi kuviksi säteenseurannaksi kutsutulla tekniikalla. Säteenseurannalla saadaan muodostettua realistisia kuvia, mutta se on laskennollisesti erittäin raskasta. Tästä syystä hahmonnuksen kohteena olevasta maisemasta muodostetaan ennen hahmonnusta avaruuden osiin jakamiseen perustuva tietorakenne, jota läpikäymällä voidaan vähentää tarvittavien laskutoimitusten määrää.

\vspace{4mm}

Avaruusjakorakenteista esitetään BSP-puu, sen erikoistapaus kd-puu ja rajaavien tilojen hierarkia BVH. Oikein muodostettuna näitä rakenteita käyttämällä voidaan vähentää hahmontamisessa tarvittavia säteiden ja monikulmioiden leikkausta testaavia operaatioita logaritmiseen määrään verrattuna niin sanottuun brute force -toteutukseen. 

\vspace{4mm}

Lopuksi vertaillaan tietorakenteita keskenään ja yritetään löytää niistä sellainen, joka nopeuttaisi säteenseurantaa eniten. Tulokset hahmontamisen nopeudessa vaihtelevat kuitenkin kuvattavien maisemien, käytettyjen implementaatioiden sekä laitteistojen välillä niin paljon, ettei parasta tietorakennetta pystytä osoittamaan.

%\vspace{4mm}

\vfill
\vspace{4mm}Asiasanat: tietokonegrafiikka, säteenseuranta, avaruusjako, BSP-puu, kd-puu, rajaava tila, BVH.

