	\documentclass[a4paper,12pt, titlepage]{article}
\usepackage{amssymb,amsthm,amsmath} %ams
\usepackage[finnish]{babel} %suomenkielinen tavutus
\usepackage[T1]{fontenc} %skanditavutus
\usepackage[utf8]{inputenc}        	% skandit utf-8 koodauksella
%\usepackage[ansinew]{inputenc}        	% skandit utf-8 koodauksella, kokeile tata, jos utf-8 ylla ei toimi.
\usepackage{cite}

\usepackage{graphicx} %dokumentti sisaltaa kuvia

\linespread{1.00} %1.24 olisi rivivali 1.5
\sloppy % Vahentaa tavutuksen tarvetta, "leventamalla" rivin keskella olevia valilyönteja.

% Lauseille, maaritelmille ja muille vastaaville voidaan maaritella omat ymparistöt
% jolloin niille saadaan yhtenainen ulkoasu
%\theoremstyle{definition}
\newtheoremstyle{break}
    {\topsep}{\topsep}%
    {\itshape}{}%
    {\bfseries}{}%
    {\newline}{}%
\theoremstyle{break}
\newtheorem{maar}{Maaritelma}[section] %Numeroidaan maaritelmat yms lukukohtaisesti. Juoksevan numeroinnin saa jattamalla [section]-option pois
\newtheorem{lause}[maar]{Lause}
\newtheorem{esimerkki}[maar]{Esimerkki}

% Algoritmeille oma tyyli
% En saanut tähän monospace fonttia. Alla on määritelty makro \code
\newtheoremstyle{algostyle}
    {\topsep}{\topsep}%
    {\normalfont}{}%
    {\bfseries}{}%
    {\newline}{}%
\theoremstyle{algostyle}
\newtheorem{alg}{Algoritmi}[section]

% Yleisimmin kayttettaville komennoille voi maaritella lyhynnemerkintöja
% esimerkiksi
\newcommand{\R}{\mathbb{R}}
\newcommand{\abs}[1]{\vert #1 \vert} % Itseisarvo
\newcommand{\tab}[1][0.5cm]{\hspace*{#1}} % Sisennys
\newcommand{\code}[1]{\texttt{#1}} % Monospace-fontti koodille
%\newcommand{\kontr}[1]{\textbf{#1}} % Boldaus ei toimi \code:n sisällä :<



\title{Avaruusjako tietokonegrafiikassa}
\author{Timo Heinonen \\LuK-tutkielma \\ tietojenkäsittelytiede \\ Turun yliopisto}
\date{Lokakuu 2016}

\begin{document}

\maketitle

\setcounter{tocdepth}{2} %sisennys
\tableofcontents

\newpage
\section{Johdanto}
\textbf{Hello World!}

\section{Tietokonegrafiikan peruskäsitteitä}
\subsection{Avaruus $\R ^3$, objektit ja polygonit} 
\subsection{Ray-Tracing tekniikka}
\begin{alg}[Ray-Tracing -algoritmi]
\code{
FOREACH pikseli $(x, y)$ näytöllä \\*
\tab FOREACH polygoni maisemassa\\*
\tab\tab ammu säde $O+t\vec{D}$ kamerasta pikselin läpi maisemaan \\*
\tab\tab IF säde osui polygoniin pisteessä $P$\\*
\tab\tab\tab laske yhteen jokaisen valonlähteen aiheuttama \\*
\tab\tab\tab varjostus pisteessä $P$  \\*
\tab\tab\tab aseta pikselin $(x, y)$ väri varjostuksen mukaisesti \\*
\tab\tab ELSE \\*
\tab\tab\tab aseta pikselin $(x, y)$ väri taustan väriseksi \\*
\tab\tab ENDIF \\*
\tab ENDFOR \\*
ENDFOR\\*}
\end{alg}

\section{Avaruusjakopuut}
\subsection{Binary Space Partitioning}
\subsection{Bounding Volume Hierarchy}

\section{Renderoinnin optimoiminen avaruusjakopuiden avulla}
\subsection{Aliavaruuspuuta käyttävä rekursiivinen Ray-Tracing algoritmi}
Viittaus\cite{ranta}, viittaus\cite{hughes} ja viittaus\cite{janke}\cite{rules}\cite{fuchs}
\cite{appel}\cite{samet}

\bibliographystyle{plain}
\bibliography{bibliography}


\end{document}
