\documentclass[a4paper,12pt, titlepage]{article}
\usepackage{amssymb,amsthm,amsmath} %ams
\usepackage[finnish]{babel} %suomenkielinen tavutus
\usepackage[T1]{fontenc} %skanditavutus
\usepackage[utf8]{inputenc}        	% skandit utf-8 koodauksella
%\usepackage[ansinew]{inputenc}        	% skandit utf-8 koodauksella, kokeile tata, jos utf-8 ylla ei toimi.
\usepackage{graphicx} %dokumentti sisaltaa kuvia
\usepackage{cite}
\usepackage[linesnumbered, boxed]{algorithm2e}
%suomennoksia yo. paketille:
\renewcommand*{\algorithmcfname}{Algoritmi}
\renewcommand*{\listalgorithmcfname}{Lista algoritmeista}


\linespread{1.00} %1.24 olisi rivivali 1.5
\sloppy % Vahentaa tavutuksen tarvetta, "leventamalla" rivin keskella olevia valilyönteja.

% Lauseille, maaritelmille ja muille vastaaville voidaan maaritella omat ymparistöt
% jolloin niille saadaan yhtenainen ulkoasu
%\theoremstyle{definition}
\newtheoremstyle{break}
    {\topsep}{\topsep}%
    {\itshape}{}%
    {\bfseries}{}%
    {\newline}{}%
\theoremstyle{break}
\newtheorem{maar}{Maaritelma}[section] %Numeroidaan maaritelmat yms lukukohtaisesti. Juoksevan numeroinnin saa jattamalla [section]-option pois
\newtheorem{lause}[maar]{Lause}
\newtheorem{esimerkki}[maar]{Esimerkki}

% Algoritmeille oma tyyli
% En saanut tähän monospace fonttia. Alla on määritelty makro \code
\newtheoremstyle{algostyle}
    {\topsep}{\topsep}%
    {\small}{}%
    {\bfseries}{}%
    {\newline}{}%
\theoremstyle{algostyle}
\newtheorem{alg}{Algoritmi}[section]

% Yleisimmin kayttettaville komennoille voi maaritella lyhynnemerkintöja
% esimerkiksi
\newcommand{\R}{\mathbb{R}}
\newcommand{\abs}[1]{\vert #1 \vert} % Itseisarvo
\newcommand{\tab}[1][0.5cm]{\hspace*{#1}} % Sisennys
\newcommand{\code}[1]{\texttt{#1}} % Monospace-fontti koodille
%\newcommand{\kontr}[1]{\textbf{#1}} % Boldaus ei toimi \code:n sisällä :<



\title{Avaruusjako tietokonegrafiikassa}
\author{Timo Heinonen \\LuK-tutkielma \\ tietojenkäsittelytiede \\ Turun yliopisto}
\date{Lokakuu 2016}

\begin{document}

\maketitle

\setcounter{tocdepth}{2} %sisennys
\tableofcontents
\listofalgorithms

\newpage
\section{Johdanto}
\textbf{Hello World!}

\section{Tietokonegrafiikan peruskäsitteitä}
\subsection{Avaruus $\R ^3$, objektit ja polygonit} 
\subsection{Ray-Tracing tekniikka}

\begin{algorithm}
\KwIn{\\\emph{Kuvataso}: $x*y$ kokoinen taulukko pikseleitä \\
\emph{Maisema}: joukko valonlähteitä ja polygoneihin jaettuja objekteja}
\KwOut{\\Kolmiulotteinen maisema projisoituna kuvatasolle}
\ForEach{pikseli $(x, y)$ näytöllä}{
  \ForEach{polygoni maisemassa}{
    Ammu säde $\vec{R} = O+t\vec{D_1}$ kamerasta pikselin läpi maisemaan \\
    \eIf{säde osui polygoniin pisteessä $P$}{
      \ForEach{Valonlähde $L$}{
	Ammu varjostussäde $\vec{R}=L-P$ valonlähdettä kohti\\
	Kasvata pisteeseen $P$ kohdistuvaa valosummaa               
      }
      Aseta pikselin $(x, y)$ väri valosumman mukaisesti
    }{
      Aseta pikseli $(x, y)$ taustan väriseksi
    }
  }
}
\caption{Ray-Tracing -algoritmi}\label{algo_raytrace}
\end{algorithm}

Ray-Tracing -tekniikan pseudokoodi on esitetty algoritmissa \ref{algo_raytrace}\\

Algoritmin suoritusta hidastaa se, että jokaista sädettä kohti on käytävä läpi kaikki maiseman polygonit ja testattava osuuko säde niihin. Säteiden ja polygonien yhteistörmäyksien määrittämiseen joudutaan joissain tapauksissa käyttämään jopa 95\% koko laskenta-ajasta.\cite{whitted} Algoritmia saataisiin siis nopeutettua huomattavasti, jos testattavien polygonien määrää jokaista sädettä kohti saataisiin vähennettyä. 

\section{Avaruusjakopuut}
\subsection{Binary Space Partitioning}
\subsection{Bounding Volume Hierarchy}

\section{Renderoinnin optimoiminen avaruusjakopuiden avulla}
\subsection{Aliavaruuspuuta käyttävä rekursiivinen Ray-Tracing algoritmi}
Viittaus\cite{ranta}, viittaus\cite{hughes} ja viittaus\cite{janke}\cite{rules}\cite{fuchs}
\cite{appel}\cite{samet}

\bibliographystyle{plain}
\bibliography{bibliography}


\end{document}
