\documentclass[a4paper,12pt, titlepage]{article}
\usepackage{amssymb,amsthm,amsmath} %ams
\usepackage[finnish]{babel} %suomenkielinen tavutus
\usepackage[fixlanguage]{babelbib}
\selectbiblanguage{finnish}

\usepackage[T1]{fontenc} %skanditavutus
\usepackage[utf8x]{inputenc}        	% skandit utf-8 koodauksella
%\usepackage[ansinew]{inputenc}        	% skandit utf-8 koodauksella, kokeile tata, jos utf-8 ylla ei toimi.
\usepackage{graphicx} %dokumentti sisaltaa kuvia
\usepackage{cite}
\usepackage[linesnumbered, boxed]{algorithm2e}
\usepackage{url}
%suomennoksia yo. paketille:
\renewcommand*{\algorithmcfname}{Algoritmi}
\renewcommand*{\listalgorithmcfname}{Lista algoritmeista}


\linespread{1.24} %1.24 olisi rivivali 1.5
\sloppy % Vahentaa tavutuksen tarvetta, "leventamalla" rivin keskella olevia valilyönteja.

% Lauseille, maaritelmille ja muille vastaaville voidaan maaritella omat ymparistöt
% jolloin niille saadaan yhtenainen ulkoasu
%\theoremstyle{definition}
\newtheoremstyle{break}
    {\topsep}{\topsep}%
    {\itshape}{}%
    {\bfseries}{}%
    {\newline}{}%
\theoremstyle{break}
\newtheorem{maar}{Maaritelma}[section] %Numeroidaan maaritelmat yms lukukohtaisesti. Juoksevan numeroinnin saa jattamalla [section]-option pois
\newtheorem{lause}[maar]{Lause}
\newtheorem{esimerkki}[maar]{Esimerkki}

% Yleisimmin kayttettaville komennoille voi maaritella lyhynnemerkintöja
% esimerkiksi
\newcommand{\R}{\mathbb{R}}
\newcommand{\abs}[1]{\vert #1 \vert} % Itseisarvo
\newcommand{\tab}[1][0.5cm]{\hspace*{#1}} % Sisennys
\newcommand{\code}[1]{\texttt{#1}} % Monospace-fontti koodille

\setlength\parindent{0pt} %uuden kappaleen sisennys

\title{Avaruusjako tietokonegrafiikassa}
\author{Timo Heinonen \\kandidaatintutkielma \\ tietojenkäsittelytiede \\ Turun yliopisto}
\date{Lokakuu 2016}

\begin{document}

\maketitle

\setcounter{tocdepth}{2} %sisennys
\tableofcontents

\newpage
\section{Johdanto}

Kolmiulotteinen tietokonegrafiikka on merkittävässä osassa animaatioelokuvissa, peleissä, sekä tietokoneavusteisessa suunnittelussa esimerkiksi arkktehtuurin ja teollisuuden alalla. Eräs aktiivinen tutkimuskohde on tietokonegrafiikan tekniikoiden soveltaminen konenäköön \cite{hughes} Tietokonegrafiikka on osin jopa syrjäyttämässä perinteistä valokuvaustyötä: huonekalujätti Ikea on siirtynyt käyttämään myyntikuvastoissaan valtaosin tietokoneella generoituja kuvia valokuvien sijaan.\cite{ikea}\\

Kolmiulotteisen grafiikan piirtämistä kaksiulotteisiksi kuviksi kutsutaan renderoinniksi. Renderoinnin lähtökohtana on kuvattava maisema (engl. \emph{scene}), joka sisältää objekteja ja valonlähteitä. Objektit ja valonlähteet on voitava mallintaa matemaattisesti, jotta niille voidaan määrittää sijainti ja suuntaus, ja niiden välisiä etäisyyksiä ja suhteita voidaan laskea. Renderointi tapahtuu aina jostakin kuvakulmasta, ja tätä varten määritellään kamera, jolla on oma sijaintinsa ja suuntauksensa maisemassa. Tämän jälkeen on selvitettävä, mitkä objektit kamera näkee, miten objekteihin osuvat valonsäteet vaikuttavat niiden väriin ja kuvan varjostukseen, ja lopuksi, mitkä värit projisoidaan kuvatason mihinkin pikseliin.\cite{janke}\\

1960-luvulla tietokonegrafiikkaa käytettiin lähinnä teollisuuden komponenttisuunnittelussa ja arkkitehtuurissa. Tietokoneella osattiin piirtää objektien ääriviivoja (engl. \emph{wireframe}), mutta varjostustekniikoita ei tunnettu. IBM:n tutkija Arthur Appel esitteli algoritmin, joka laski suoran yhtälöitä, eli säteitä kuvasta maisemaan ja siitä valonlähteisiin. Tämän tekniikan avulla voitiin piirtää yksinkertaisia varjostuksia, ja myöhemmin tästä Ray Tracing:iksi kutsutusta tekniikasta tuli erittäin suosittu.\cite{appel}\\

Jo Appel totesi Ray Tracing -tekniikan olevan erittäin aikaavievä.\cite{appel} Vaikka tietokoneiden, ja varsinkin grafiikkaprosessoreiden laskentateho kasvaa jatkuvasti, ei grafiikan tuottaminen ole vieläkään halpaa tai nopeaa. Kuvista halutaan jatkuvasti fotorealistisempia, ja yksityiskohtaisemmat kuvattavat mallit ja monimutkaiset valaisutekniikat vaativat erittäin paljon laskentatehoa. Esimerkiksi elokuvastudio Pixarin Monsterit-yliopisto -animaatioelokuvan piirtäminen vaati yli sata miljoonaa prosessorituntia.\cite{monsterit} Tämän takia renderoinnin nopeuttaminen on ollut aktiivinen tutkimuskohde jo pitkään.\\

1980-luvulla kehitettiin tekniikoita, joilla voitiin vähentää säteiden ja maiseman osumatarkasteluiden määrää. Steven Rubin ja Turner Whitted esittelivät tekniikan, jossa maisema jaetaan esiprosessointivaiheessa hierarkisesti suuntaissärmiöihin ihmisen toimesta. Säteiden ja suuntaissärmiöiden osumia tarkastelemalla voitiin vähentää operaatioiden kokonaismäärää.\cite{rubin} Henry Fuchsin et al. metodiin kuului myös esiprosessointivaihe, tällä kertaa tietokoneen suorittamana. Maiseman objektit oli jaettu pienempiin osiin, polygoneihin, joista valittiin binääripuun juureksi sopiva. Tämän jälkeen binääripuuhun lisättiin polygoneja sillä perusteella, onko niiden vanhempi puussa etu- vai takapuolella. Näillä tekniikoilla renderointia saatiin nopeutettua siirtämällä työtä esiprosessointivaiheeseen.\cite{fuchs}  \\

Tässä tutkielmassa esitellään avaruuden jakamiseen perustuvia tietorakenteita, joilla kolmiulotteisten kuvien renderointia voidaan nopeuttaa. Luvussa 2 määritellään joitakin grafiikan peruskäsitteitä, sekä esitetään algoritmi Ray Tracing -tekniikalle. Luvussa 3 tutkitaan Binary Space Partitioning -puuta, kd-puuta ja Bounding Volume -hierarkiaa, sekä niiden rakentamiseen ja läpikäyntiin liittyviä algoritmeja. Luvussa 4 selvitetään, miten kd-puuta voidaan käyttää optimoimaan Ray Tracing -algoritmia.


\newpage
\section{Kolmiulotteisen tietokonegrafiikan peruskäsitteitä}
\subsection{Määritelmiä}

Kolmiulottesten kuvien renderoinnin kohteena ovat \emph{objektit}, jotka mallintavat jotakin esinettä tai muotoa avaruudessa $\R^3$. Objektit voidaan esittää tietokoneen muistissa taulukkona pisteitä $(x,y,z)\,\epsilon\,\R^3$: esimerkiksi kolmiota voidaan kuvata kolmella pisteellä, ja palloa kahdella pisteellä, jotka esittävät sen keskipistettä ja yhtä pistettä sen pinnalla.\cite{angel}\\

Objektit jaetaan lähes kaikissa ei-triviaaleissa tapauksissa \emph{polygoneihin}. Polygoni on samassa tasossa olevilla kärjillä rajattu alue, jonka kärkien muodostamat janat eivät leikkaa toisiaan. Useimmiten grafiikkasovelluksissa ja -rajapinnoissa valitaan polygonien muodoksi kolmiot, sillä niiden kolme kärkeä muodostavat aina tason, ja grafiikkaprosessorit osaavat operoida kolmioilla erittäin nopeasti.\cite{angel}\\

Jotta voitaisiin tarkastella objektien, valonlähteiden, ja kuvakulman, eli \emph{kameran} välisiä suhteita ja suuntauksia avaruudessa, valitaan kolme koordinaatistoa, jotka on määritelty kolmella toisiinsa nähden kohtisuorilla kantavektoreilla. \emph{Lokaalikoordinaatisto} sisältää yksittäisen objektin geometrian. Useimmitien origo sijoitetaan objektin keskipisteeseen. \emph{Maailma-koordinaatisto} kuvaa koko avaruutta, ja sisältää tietoa siitä, mihin objektien lokaalikoordinaatistojen origot on sijoitettu. Lopuksi tämä maisema kuvataan virtuaalisella kameralla, jolla on oma \emph{kamerakoordinaatistonsa}. Koordinaatistosta toiseen siirtyminen, koordinaatistojen skaalaus ja rotaatio voidaan toteuttaa lineaarikuvauksilla.\cite{janke}

\subsection{Ray-Tracing -tekniikka}

Ray Tracing on renderointitekniikka, jolla voidaan piirtää erittäin fotorealistisia kuvia. Ray Tracing -tekniikka pyrkii mallintamaan valonsäteitä, jotka saavat alkunsa valonlähteistä, kulkevat avaruudessa ja osuvat objekteihin valaisten niitä, kimmoten niistä toisiin objekteihin ja muodostaen varjoja. Jotkut valonsäteet lopulta löytävät tiensä katsojan silmiin, eli kameraan. Koska olisi mahdotonta selvittää jokaisen valonsäteen kulkua avaruudessa, Ray Tracing -algoritmi ottaa huomioon vain ne säteet, jotka todella osuvat kameraan. Valonsäteitä seurataan siis käänteisessä järjestyksessä, kamerasta objekteihin, ja niistä valonlähteisiin.\cite{janke}\\

Ray Tracing -algoritmi muodostaa sille syötteenä annetusta kolmiulotteisesta maisemasta kameran sijainnin perusteella kaksiulotteisen kuvan. Jokaisen kuvatason pikselin läpi ammutaan säde $\vec{R}=O+t\vec{D}$, missä $O$ on kameran sijainti maailma-koordinaatistossa, $D$ kuvaa säteen kulkusuuntaa ja $t\,\epsilon\,\R$. Säteellä etsitään törmäyspistettä lähimmän objektin kanssa, eli sellaista mahdollisimman pientä arvoa $t$, että piste $P=O+t$ on jonkin objektin pinnalla. Tällöin osuman saaneen objektin piste $P$ voi näkyä kameraan, mikäli siihen osuu valoa. Osumakohdasta ammutaan uusi, varjostussäteeksi kutsuttu säde. Jos varjostussäde osuu suoraan, tai kimmoten muista objekteista valonlähteeseen, lankeaa objektin pinnalle valoa. Tekniikan pseudokoodi on esitetty algoritmissa \ref{algo_raytrace}.\cite{janke}\\

\begin{algorithm}
\KwIn{\\$Kuvataso$: $x*y$ kokoinen taulukko pikseleitä \\
$Maisema$: joukko valonlähteitä ja polygoneihin jaettuja objekteja}
\KwOut{\\Kolmiulotteinen maisema projisoituna kuvatasolle}
\ForEach{kuvatason pikseli $(x,y)$}{
  $etaisyys \gets \infty$ \\
  \ForEach{polygoni maisemassa}{
    Ammu säde $\vec{R}=O+t\vec{D}$ kamerasta pikselin läpi maisemaan \\
    \eIf{säde osui polygoniin pisteessä $P$ {\bf and} t < etäisyys}{
      	Valon määrä $V \gets 0$\\
	\ForEach{valonlähde $L$}{
	Ammu varjostussäde $\vec{R}=L-P$ valonlähdettä kohti\\
	Kasvata valosummaa $V$               
      }
      Aseta pikselin $(x,y)$ väri valosumman $V$ mukaisesti
    }{
      Aseta pikseli $(x,y)$ taustan väriseksi
    }
  }
}
\KwRet $kuvataso$
\caption{Ray-Tracing -algoritmi}\label{algo_raytrace}
\end{algorithm}



Algoritmin suoritusnopeutta rajoittaa se, että jokaista sädettä kohti on käytävä läpi kaikki maiseman polygonit ja testattava osuuko säde niihin. Säteiden ja polygonien leikkauksien määrittämiseen joudutaan joissain tapauksissa käyttämään jopa 95\% koko laskenta-ajasta.\cite{whitted} Algoritmia saataisiin siis nopeutettua huomattavasti, jos testattavien polygonien määrää jokaista sädettä kohti saataisiin vähennettyä. Yleisesti käytetty tapa leikkaustestien vähentämiseksi on muodostaa maisemasta ennen varsinaista renderointia hierarkinen tietorakenne, jota läpikäymällä saavutetaan nopeasti säteen leikkaama polygoni.\cite{rubin}

\section{Avaruusjakopuut}
\subsection{BSP-puu}
\subsection{kd-puu}
\subsection{Bounding Volume Hierarchy}

\section{Renderoinnin optimoiminen avaruusjakopuiden avulla}
\subsection{kd-puuta käyttävä Ray-Tracing -algoritmi}
Viittaus\cite{ranta},\cite{samet}\cite{havran}

\newpage
\bibliographystyle{apalike} %bibplain vai apalike?
\bibliography{bibliography}
\listofalgorithms

\end{document}
