\documentclass[a4paper,12pt, titlepage]{article}
\usepackage{amssymb,amsthm,amsmath} %ams
\usepackage[finnish]{babel} %suomenkielinen tavutus
\usepackage[T1]{fontenc} %skanditavutus
\usepackage[utf8]{inputenc}        	% skandit utf-8 koodauksella
%\usepackage[ansinew]{inputenc}        	% skandit utf-8 koodauksella, kokeile tata, jos utf-8 ylla ei toimi.

\usepackage{graphicx} %dokumentti sisaltaa kuvia

\linespread{1.24} %rivivali 1.5
\sloppy % Vahentaa tavutuksen tarvetta, "leventamalla" rivin keskella olevia valilyönteja.

% Lauseille, maaritelmille ja muille vastaaville voidaan maaritella omat ymparistöt
% jolloin niille saadaan yhtenainen ulkoasu
%\theoremstyle{definition}
\newtheoremstyle{break}
    {\topsep}{\topsep}%
    {\itshape}{}%
    {\bfseries}{}%
    {\newline}{}%
\theoremstyle{break}
\newtheorem{maar}{Maaritelma}[section] %Numeroidaan maaritelmat yms lukukohtaisesti. Juoksevan numeroinnin saa jattamalla [section]-option pois
\newtheorem{lemma}[maar]{Lemma} % Tassa ymparistö 'lemma' kayttaa laskuria 'maaritelma'
%\newtheorem{lemma}{Lemma}[section] %Ymparistölle 'lemma' voitaisiin maaritella myös oma laskurinsa.
\newtheorem{lause}[maar]{Lause}
\newtheorem{esimerkki}[maar]{Esimerkki}

% Algoritmeille oma tyyli
% En saanut tähän monospace fonttia. Alla on määritelty makro \code
\newtheoremstyle{algostyle}
    {\topsep}{\topsep}%
    {\normalfont}{}%
    {\bfseries}{}%
    {\newline}{}%
\theoremstyle{algostyle}
\newtheorem{alg}{Algoritmi}[section]

% Yleisimmin kayttettaville komennoille voi maaritella lyhynnemerkintöja
% esimerkiksi
\newcommand{\Q}{\mathbb{Q}}
\newcommand{\R}{\mathbb{R}}
\newcommand{\Z}{\mathbb{Z}}
\newcommand{\C}{\mathbb{C}}
\newcommand{\abs}[1]{\vert #1 \vert} % Itseisarvo
\newcommand{\tab}[1][1cm]{\hspace*{#1}} % Sisennys
\newcommand{\code}[1]{\texttt{#1}} % Monospace-fontti koodille
%\newcommand{\kontr}[1]{\textbf{#1}} % Boldaus ei toimi \code:n sisällä :<



\title{Kolmiulotteisen avaruuden jakaminen aliavaruuksiin}
\author{Timo Heinonen \\LuK-tutkielma \\ tietojenkäsittelytiede \\ Turun yliopisto}
\date{Lokakuu 2016}

\begin{document}

\maketitle

\setcounter{tocdepth}{2} %sisennys
\tableofcontents

\newpage
\section{Johdanto}
\textbf{Hello World!}

\section{Tietokonegrafiikan peruskäsitteitä}
\subsection{Avaruus $\R ^3$ ja objektit} 
\subsection{Näkyvyysongelma}
\subsection{Ray-Tracing tekniikka}
\begin{alg}[Ray-Tracing -algoritmi]
\code{
FOREACH pikselinäytöllä \\*
\tab FOREACH renderointiprimitiivi \\*
\tab\tab ammu säde $O+tD$ kamerasta pikselin läpi avaruuteen \\*
\tab\tab IF säde osui primitiiviin \\*
\tab\tab\tab selvitä varjostus, valon heijastuminen, yms. \\*
\tab\tab\tab aseta pikselin väri primitiivin väriseksi \\*
\tab\tab ELSE \\*
\tab\tab\tab aseta pikselin väri taustan väriseksi \\*
\tab\tab ENDIF \\*
\tab ENDFOR \\*
ENDFOR\\*}
\end{alg}

\begin{lause}[Ray-Tracing algoritmin kompleksisuus]
Sehän on: $O($se ny riippuu...$)$ 
\end{lause}

\section{Aliavaruushierarkiat}
\subsection{Binary Space Partitioning}
\textit{BSP}
\subsection{Bounding Volume Hierarchy}
\textit{BVH}

\section{Renderoinnin optimoiminen aliavaruuksien avulla}
\subsection{Aliavaruuspuuta käyttävä rekursiivinen Ray-Tracing algoritmi}


\end{document}
